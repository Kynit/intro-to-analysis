\documentclass[answers]{exam}
\begin{document}
\firstpageheader{}{}{\bf\large Assignment 0 \\ Greg d'Eon \\ }
\runningheader{Name}{Class Assignment}{Due Date}

\begin{questions}
\question %Q1

There are three boxes of candy. One box contains mint candies, one chocolate candies, and the other is mixed. All three boxes are incorrectly labeled. What is the smallest number of candies that you need to remove and sample to be able to correctly label all three boxes?

\begin{solution} 
\textbf{1 sample}

	First, we note that there are only two derangements of the three labels:
	
	\begin{center}
		\begin{tabular}{c | c | c | c }
			Label & Mint & Chocolate & Mixed \\
			\hline
			(1) & Chocolate & Mixed & Mint \\
			(2) & Mixed & Mint & Chocolate \\
		\end{tabular}
	\end{center}
	
	We notice that the box labeled `mixed' must contain only one type of candy. The solution is to take a single candy from the `mixed' box and to match the sampled candy with one of the two orders above. This process takes only one candy sample.
\end{solution}

\question %Q2
A sealed room contains one lightbulb. Outside of the room, there are three switches, only one of which operates the bulb. You are outside the room, able to operate the switches in any way you see fit, but when the door is opened for the first time, you must determine which switch operates the light. What do you do?

\begin{solution}
(unsolved)
\end{solution}

\question %Q3
Describe how one can use a four-minute hourglass and a seven-minute hourglass to measure a period of nine minutes.

\begin{solution}
\textbf{13 mins - 4 mins = 9 mins}

First, notice that both hourglasses should always be active - if only one is running at any time, the next notable time will occur when both hourglasses are empty, which is equivalent to a restart.

Then, it is possible to brute-force search for a sequence that has a 9 minute gap. Here is one, showing the time remaining in each hourglass at each flip:

\begin{center}
	\begin{tabular}{c | c | c}
		Time & 4 minute & 7 minute \\
		\hline
		0 & 0 & 0 \\
		  & 4 & 7 \\
		\hline
		4 & 0 & 3 \\
		  & 4 & 3 \\
		\hline
		7 & 1 & 0 \\
		  & 3 & 7 \\
		\hline
		10& 0 & 4 \\
		  & 0 & 3 \\
		\hline
		13& 0 & 0 \\
	\end{tabular}
\end{center}

From this table, we can see that there is a 9 minute gap from 4 minutes to 13 minutes.

\end{solution}

\question %Q4
Two doors are guarded by two men, one of whom always lies and one of whom always tells the truth; however, you do not know which man is which. One of the doors leads to freedom and one to captivity. Determine a single question that, if asked of one of the guards, would reveal the door to freedom with certainty.

\begin{solution}
\textbf{``If I asked the other guard which door leads to freedom, which one would they reveal?''} (and take the other one)
\end{solution}

\question %Q5
A cruel calculus instructor decided to terrorize her students. The instructor announced that during the next class, the students will line up, facing away from the front of the line. The instructor will then place either a white or a gray dunce cap on each student's head. Each student will be unable to see his or her own cap but will be able to see the cap colors of all those classmates who are in front of him or her.

Starting at the head of the line, each student will be asked, in turn, ``What is the color of your dunce cap?'' Students will only be allowed to respond by saying ``white'' or ``gray.'' The students who answer correctly will be given As and the students who answer incorrectly will fail the course. For each student, the instructor will respond with either ``Correct! You receive an A'' or ``you fail! Get out of my classroom, you dunce,'' depending on the correctness of the answer. All students will be able to hear all students' answers and the instructor's responses. Knowing this horrific fate that awaits them, the students have all night to come up with a plan. If there are $n$ students in the class, how many of them can be guaranteed to receive As? Your challenge is to devise a scheme that the students can employ to allow as many of them as possible to receive As.

\begin{solution}
(unsolved)
\end{solution}

\question %Q6
Consider the following mathematical illusion: A regular deck of 52 playing cards is shuffled several times by an audience member until everyone agrees that the cards are completely shuffled. Then, without looking at the cards themselves, the magician divides the deck into two equal piles of 26 cards. The magician taps both piles of face-down cards three times. Then, one by one, the magician reveals the cards of both piles. Magically, the magician is able to have the cards arrange themselves so that the number of cards showing black suits in the first pile is identical to the number of cards showing red suits in the second pile. Your challenge is to figure out that secret to this illusion and then perform it to your friends.

\begin{solution}
\textbf{Act normal.}

A deck of 52 cards has 26 black cards and 26 red cards. Suppose that these cards are randomly distributed into two piles called A and B, each with 26 cards. If $B_A$ is the number of black cards in pile A, then the number of black cards in pile B is 
\begin{eqnarray*}
B_B &=& 26 - B_A
\end{eqnarray*}
and since the remaining cards in pile B are red, the number of red cards in pile B is
\begin{eqnarray*}
R_B &=& 26 - B_B \\
	&=& 26 - (26 - B_A) \\
	&=& B_A
\end{eqnarray*}
so the number of black cards in one pile always match the number of red cards in the other. No magic is required.
\end{solution}

\question %Q7
Some number of coins are spread out on a table. They lie either heads up or tails up. Unfortunately, you are blindfolded and thus both the coins and the table upon which they sit are hidden from view. Certainly you can feel your way across the table and count the total number of coins on the table's surface, but you cannot determine if any individual coin rests heads up or down (perhaps you are wearing gloves). You are informed of one fact (beyond the number of coins on the table): Someone tells you the number of coins that are lying heads up. You can now rearrange the coins, turn any of them over, and move them in any way you wish, as long as the final configuration has all the coins resting (heads or tails up) on the table. Your challenge is to turn over whatever coins you wish and divide the coins into two collections so that one collection of coins contains the same number of heads up coins as the other collection contains.
\begin{solution}
Suppose that we have $n$ coins with $h$ of them showing heads. If we randomly make a group of $h$ coins, this group of coins will contain $i$ heads-up coins, where $0 \leq i \leq h$, leaving $h - i$ coins in the other group. Since there are $h - i$ tails-up coins in the first group, flipping the entire first group will leave us with two groups containing an equal number of heads.
\end{solution}

\question %Q8
You find yourself on a reality TV show that has you competing with other real people in totally artificial circumstances. In one scenario, you are given nine balls of clay You are informed by the program's B-celebrity host that hidden inside of one of those clay balls is a keey that will unlock a refrigerator that houses a vast quantity of food. Since the producers ``thought'' the ratings would be higher if the contestants were deprived of nutrition, even the thought of brussels sprouts makes your mouth water. You are told that the eight balls that do not contain the key to your dietary dreams all weight the same. The special ball with the key inside weighs slightly more, but not enough for you to feel the difference by holding the balls in your hand. One of the program's sponsors, \textit{Replace-Oh!}, the company that manufactures one-time-use balance scales (with the slogan ``Weight aweigh then throw away!''), has agreed to provide some of its scales in exchange for a few shameless plugs throughout the program. Their scales will tell which side is heavier and then instantly self-destruct. You are only allowed to break open one clay ball to see if you can find the refrigerator key. Your challenge is to determine the fewest disposable balance scales required to guarantee that you can identify the ball with the key. Justify your answer.
\begin{solution}
\textbf{2 scales are required.}

First, we can split the 9 balls into 3 groups of 3 balls. Then, we can use 1 scale to weigh 2 of these groups against each other. If one is heavier, then it must have the key; if neither is heavier, then the unweighed group has the key. Then, we can take the selected group and weigh two of the balls against each other. Using the same selection process, we can find the key with this second measurement.
\end{solution}

\end{questions}
\end{document}
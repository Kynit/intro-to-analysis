\documentclass{article}
%\setlength\parindent{0pt}

\usepackage{amsthm}

\theoremstyle{definition}
\newtheorem{problem}{Problem}

\begin{document}
\section*{Problem Sequence - Problems}
This document contains the problems that we are given throughout the course.

\begin{problem}
Show that if $M$ is the set which contains only the number $0$, and $p$ is a point, then $p$ is not a limit point of $M$.
\end{problem}

\begin{problem}
Show that if $M$ is the point set that contains only the two points $0$ and $1$, then no point is a limit point of $M$.
\end{problem}

\begin{problem}
Show that $1$ is a limit point of the segment $(0, 1)$.
\end{problem}

\begin{problem}
Show that $a$ and $b$ are limit points of the segment $(a, b)$.
\end{problem}

\begin{problem}
If $S$ is the segment $(a, b)$, show that every point of $S$ is a limit point of $S$.
\end{problem}

\begin{problem}
Show that if $M$ is an interval and $p$ is a point not in $M$, then $p$ is not a limit point of $M$.
\end{problem}

\begin{problem}
Show that if $H$ is a point set, and $K$ is a point set, and $H \subseteq K$, and $p$ is a limit point of $H$, then $p$ is a limit point of $K$.
\end{problem}

\begin{problem}
Show that if $M$ is the set of all positive integers and $p$ is a point, then $p$ is not a limit point of $M$.
\end{problem}

\begin{problem}
Show that if $p$ is a limit point of the point set $M$ and $S$ is a segment containing $p$, then $S$ contains $2$ points of $M$.
\end{problem}

\begin{problem}
Let $H$ be a point set which has a limit point, and let $K$ be the set of all limit points of $H$. Show that if $p$ is a limit point of $K$, then $p$ is also a limit point of $H$.
\end{problem}

\begin{problem}
Show that if $M$ is the set of all reciprocals of positive integers, then $0$ is a limit point of $M$.
\end{problem}

\begin{problem}
Show that if $p \neq 0$, then $p$ is not a limit point of the set of all reciprocals of positive integers.
\end{problem}

\begin{problem}
if $H$ and $K$ are two point sets having a common point, and $p$ is a limit point of $H \cap K$, then $p$ is a limit point of $H$ and $p$ is a limit point of $K$.
\end{problem}

\begin{problem}
Suppose that $M$ is a point set and $p$ is a limit point of $M$. Must it be true that every interval containing $p$ contains a point of $M$ different from $p$?
\end{problem}

\begin{problem}
Suppose that $M$ is a point set and every interval containing $p$ contains a point of $M$ different from $p$. Must it be true that $p$ is a limit point of $M$?
\end{problem}

\begin{problem}
Show that if the point $p$ is in each of the two segments $S_1$ and $S_2$, then $S_1 \cap S_2$ is a segment containing $p$.
\end{problem}

\begin{problem}
Show that if $p$ is not a limit point of the point set $H$ and $p$ is not a limit point of the point set $K$, then $p$ is not a limit point of $H \cup K$.
\end{problem}

\begin{problem}
Show that if $H$ and $K$ are two point sets and $p$ is a limit point of $H \cup K$, then $p$ is a limit point of $H$ or $p$ is a limit point of $K$.
\end{problem}

\begin{problem}
Show that if $M$ is a point set and there is no smallest number in $M$ (or there is no largest number in $M$), then if $n$ is a positive integer, $M$ has $n$ points so that $M$ is infinite.
\end{problem}

\begin{problem}
Show that if $M$ is a point set and $M$ is finite then there is a smallest (and a largest) number in $M$.
\end{problem}

\begin{problem}
Show that if $M$ is a point set, and $p$ is a limit point of $M$, and $S$ is a segment containing $p$, and $n$ is a positive integer, then $S$ contains $n$ points so that $S \cap M$ is infinite.
\end{problem}

\begin{problem}
What does it mean to say that a sequence $x_1, x_2, x_3, \dots$ does not converge to the number $c$?
\end{problem}

\begin{problem}
Show that the sequence $0, 1, 0, 1, \dots$ does not converge to $0$.
\end{problem}

\begin{problem}
Show that the sequence $1, \frac{1}{2}, \frac{1}{3}, \frac{1}{4}, \dots$ converges to $0$.

(not assigned as course work)
\end{problem}

\begin{problem}
Show that if the sequence $x_1, x_2, x_3, \dots$ converges to the number $c$ and $(a, b)$ is a segment containing $c$ then there are not infinitely many terms of the sequence $x_1, x_2, x_3, \dots$ that are not in $(a, b)$.
\end{problem}

\begin{problem}
Does the sequence $\frac{1}{2}, \frac{1}{1}, \frac{1}{4}, \frac{1}{3}, \frac{1}{6}, \frac{1}{5}, \dots$ converge to $0$? Note that this is the sequence $x$ where $x_n = \frac{1}{n-1}$ if $n$ is an even positive integer and $x_n = \frac{1}{n+1}$ if $n$ is an odd positive integer.
\end{problem}

\begin{problem}
Assume that the sequence $x_1, x_2, x_3, \dots$ converges to the point $c$ and $d$ is a point different from $c$. Show that $x_1, x_2, x_3, \dots$ does not converge to $d$.
\end{problem}

\begin{problem}
Show that if the sequence $x_1, x_2, x_3, \dots$ converges to the point $c$, and, for each positive integer $n$, $x_n \neq x_{n+1}$, then $c$ is a limit point of the range of the sequence.
\end{problem}

\begin{problem}
Find a sequence $x_1, x_2, x_3, \dots$ such that the point $c$ is a limit point of the range $\{x_1, x_2, \dots \}$ of the sequence but $x_1, x_2, x_3, \dots$ does not converge to $c$. 
\end{problem}

\begin{problem}
Show that if $d$ is a number and $x_1, x_2, x_3, \dots$ is a sequence which converges to the point $c$, then the sequence $d \cdot x_1, d \cdot x_2, d \cdot x_3, \dots$ converges to $d \cdot c$.
\end{problem}

\begin{problem}
Find examples of infinite sets, one having a first point and one not having a first point.
\end{problem}

\begin{problem}
Assume that $x_1, x_2, x_3, \dots$ is a sequence that converges to the point $c$ and $d$ is a point different from$c$. Show that $d$ is not a limit point of the range of the sequence.
\end{problem}

\begin{problem}
Show that if $M$ is a point set, there cannot be both a rightmost point of $M$ and a smallest number which is larger than each number in $M$.
\end{problem}

\begin{problem}
If $M$ is a point set and each point of $M$ is to the left of $p$ and $p$ is the smallest number such that each point of $M$ is to the left of $p$, then $p$ is a limit point of $M$.
\end{problem}

\begin{problem}
If the range of the increasing sequence $x_1, x_2, x_3, \dots$ is bounded above, then $x_1, x_2, x_3, \dots$ converges to some point.
\end{problem}

\begin{problem}
Assume that $x_1, x_2, x_3, \dots$ is a sequence and $c$ is a point. Suppose that, given any positive number $\epsilon$, we can find an integer $n$ such that $x_n, x_{n+1}, x_{n+2}, \dots$ are in $(c - \epsilon, c + \epsilon)$. Show that $x_1, x_2, x_3, \dots$ converges to $c$.
\end{problem}

\begin{problem}
Show that if the sequence $p_1, p_2, p_3, \dots$ converges to $c$ and the sequence $q_1, q_2, q_3, \dots$ converges to $d$, then the sequence $p_1 + q_1, p_2 + q_2, p_3 + q_3, \dots$ converges to $c + d$.
\end{problem}

\begin{problem}
If the range of the non decreasing sequence $x_1, x_2, x_3, \dots$ is bounded above, then $x_1, x_2, x_3, \dots$ converges to some point.
\end{problem}

\begin{problem}
If $M$ is a bounded point set then the lub($M$) (and the glb($M$)) is either a point of $M$ or a limit point of $M$.
\end{problem}

\begin{problem}
Give an example of a bounded point set such that lub($M$) is both a point of $M$ and a limit point of $M$.
\end{problem}

\begin{problem}
If the sequence $x_1, x_2, x_3,\dots$ converges to the point $c$, then $M = {x_1, x_2, x_3, \dots}$ is bounded.
\end{problem}

\begin{problem}
If $M$ is an infinite and bounded point set, then $M$ has a limit point.
\end{problem}

\begin{problem}
Show that if $f$ is the graph which contains only the two points $(0,0)$ and $(1,1)$, then $f$ is continuous at the point $(0,0)$.
\end{problem}

\begin{problem}
Let $f$ be the graph such that $f(x) = x^2$ for each number $x$. Show that $f$ is continuous at the point $(2, 4)$.
\end{problem}

\begin{problem}
Let $f$ be the simple graph such that $f(x) = 1$ for all numbers $x > 0$, and $f(0) = 0$. Show that $f$ is not continuous at the point $(0, 0)$.
\end{problem}

\begin{problem}
If $f$ is a simple graph and $x_1, x_2, x_3, \dots$ is a sequence of points in the domain of $f$ converging to the number $c$ in the domain of $f$, and $f$ is continuous at $(c, f(c))$, then $f(x_1), f(x_2), f(x_3), \dots$ converges to $f(c)$.
\end{problem}

\end{document}
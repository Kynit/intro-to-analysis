\documentclass{article}

\usepackage{amsthm}
\theoremstyle{definition}

\newtheorem{solution}{Solution}

\begin{document}
\section*{Problem Sequence - Solutions}
This document will be filled up with the solutions from the problem sequence.

\begin{solution} %1
(Greg)

From Axiom 1, there are three cases to consider:
\begin{enumerate}
\item
$p = 0$

$p$ is equal to the only element of $M$. 
From the definition of limit point, every segment containing $p$ must contain a point of $M$ different from $p$. 
However, there are no points in $M$ different from $p$, so $p$ must not be a limit point of $M$.

\item
$p > 0$

From Axiom 3, there exists a point $a$ such that $0 < a < p$. 
There also exists a point $b$ such that $b > p$. 
Since $p > 0$ and $b > p$, Axiom 2 tells us that $b > 0$. 
We can then form the segment $S = (a, b)$. 
Since $a > 0$ and $b > 0$, $0$ is not between $a$ and $b$, so $S$ does not contain $0$. 
However, $a < p < b$, so $S$ contains $p$. 
$S$ is a segment containing $p$ that does not contain any element of $M$, so $p$ is not a limit point of $M$.

\item
$p < 0$

(symmetric to the $p > 0$ case)
\end{enumerate}

Therefore, regardless of our choice of $p$, we can construct a segment that contradicts the requirements in the limit point definition, so $p$ is not a limit point of $M$.
\end{solution}

\begin{solution} %2
(Jeff)

According to Definition 4, we can prove that $p$ is not a limit point of $M$ if we can construct a segment containing $p$ but not a different point of $M$.
Construct this segment as follows:

\begin{enumerate}
\item
If all points of $M$ are on the opposite side of $p$, choose any value as an endpoint.

\item
If any point of $M$ is on the same side of $p$, choose a point between the nearest point of $M$ and $p$.
(Axiom 3 confirms that there will be such a point.)
\end{enumerate}

This segment contains $p$ but will not contain any points of $M$ (with the exception of $p$, if $p$ is $0$ or $1$).
Therefore, we have found a segment that does not fulfill the requirements of Definition 4, so $p$ is not a limit point of $M$.
\end{solution}

\begin{solution} %3
(Erin, solved after Problem 4)

Let $a = 0$ and $b = 1$. Then, according to Problem 4, $b$ is a limit point of $(a, b)$. Therefore, $1$ is a limit point of $(0,1)$.
\end{solution}

\begin{solution} %4
(Erin)

We will prove that $b$ is a limit point of $(a, b)$; The proof for $a$ is similar.

Construct a segment $(p, q)$ containing $b$ ($p < b < q$).
According to Axiom 1, there are three cases, and we will deal with two of them simultaneously:

\begin{enumerate}
\item $p < a$

In this case, $p < a < b < q$, so $p < (a, b) < q$.
Since $(p, q)$ contains every point of $(a, b)$, we have found points that satisfy Defintion 4.

\item $p \geq a$

In this case, $a \leq p < b < q$.
According to Axiom 3, there is a point $d$ between $p$ and $b$. 
The inequality is then $a \leq p < d < b < q$.
Then,

\begin{itemize}
\item $a < d < b$, so $d$ is in $(a, b)$.
\item $p < d < q$, so $d$ is in $(p, q)$.
\end{itemize}

This means that $(p, q)$ contains $d$, which is a point of $(a, b)$.
\end{enumerate}

Therefore, every $(p, q)$ containing $b$ also contains a point of $(a, b)$, so $b$ is a limit point of $(a, b)$.
\end{solution}

\begin{solution} %5
(Greg)

Choose any point $p$ from $S$.
Construct a segment $(x, y)$ that contains $p$ (ie: $x < p < y$).
Put no other condition on $y$.
According to Axiom 1, one of these three cases is true:

\begin{enumerate}
\item 
$x > a$

Choose a point $q$ between $x$ and $p$ ($x < q < p$; this exists by Axiom 3).
Apply Axiom 2 three times:
\begin{itemize}
\item $a < x$ and $x < q$, so $a < q$
\item $q < p$ and $p < b$, so $q < b$
\item $q < p$ and $p < y$, so $q < y$
\end{itemize}

so $a < q < b$ and $x < q < y$.
Therefore, $q$ is an element of $S$ inside $(a, b)$ that is different from $p$.

\item
$x = a$

Repeat the proof for $x > a$ with one change:

\begin{itemize}
\item $a = x$ and $x < q$, so $a < q$
\end{itemize}

\item
$x < a$

Choose $q$ betweem $a$ and $p$.
Change:

\begin{itemize}
\item $x < a$ and $a < q$, so $x < q$
\end{itemize}

(same conclusion as $x > a$)
\end{enumerate}

In each of these three cases, every $(x,y)$ contains a point $q$ from $(a, b)$.
Therefore, we have satisfied Definition 4, so $p$ is a limit point of $(a, b)$.

\end{solution}

\begin{solution} %6
\end{solution}

\begin{solution} %7
(Amber)

\end{solution}

\begin{solution} %8
(Greg)

Consider a point $p$.
According to Axiom 4, there exists a largest integer $M_x$ and a smallest integer $M_y$ such that $M_x < p < M_y$.
Then, choose points $x$ and $y$ from Axiom 3 such that $M_x < x < p$ and $p < y < M_y$, and consider the segment $S = (x, y)$.

We will try to find an integer different from $p$ inside $S$.
Axiom 1 gives us three cases:

\begin{enumerate}
\item $n = p$ (note: this is only possible if $p$ is an integer)

Here, $n$ is not different from $p$, so we have not found an integer different from $p$.

\item $n < p$

Note that $n$ must satisfy $n \leq M_x$; if $n > M_x$, we have contradicted Axiom 4.
Then, from Axiom 2, $n \leq M_x$ and $M_x < x$, so $n < x$.
This shows that $n$ is not between $x$ and $y$, so $S$ does not contain $n$.

\item $n > p$

This case is symmetric to the $n < p$ case.

\end{enumerate}

Therefore, $S$ is a segment containing $p$ that does not contain any integers different from $p$.
Since every element of $M$ is an integer, we have proven that there exists a segment $S$ for every point $p$ without any other points of $M$, so $M$ has no limit points.

\end{solution}

\begin{solution} %9
\end{solution}

\begin{solution} %10
\end{solution}

\begin{solution} %11
(Greg)

Construct a segment $S = (x, y)$ that contains $0$ ($x < 0 < y$).
Since the reciprocal of a positive number is positive, every element $M_i$ of $M$ is positive, so $x < M_i$ from Axiom 2.

Then, we will attempt to find an element of $M$ that is less than $y$.
Since every element of $M$ is of the form $\frac{1}{n}$ for some integer $n$, we are looking for $\frac{1}{n} < y$.
From the properties of reciprocals, this is equivalent to $\frac{1}{y} < n$.
According to Axiom 4, the point $\frac{1}{y}$ has an integer greater than it, so we can find an $n$ that satisfies this inequality.

Therefore, every $S$ containing $0$ also contains a point of $M$, so $0$ is a limit point of $M$.
\end{solution}

\begin{solution} %12
\end{solution}

\begin{solution} %13
\end{solution}

\begin{solution} %14
\end{solution}

\begin{solution} %15
\end{solution}

\begin{solution} %16
(Fernando)

Let $S_1 = (a_1, b_1)$ and $S_2 = (a_2, b_2)$.
Since $p$ is in both segments, $a_1 < p < b_1$ and $a_2 < p < b_2$.
From Axiom 2, we know that:

\begin{itemize}
\item $a_1 < b_1$
\item $a_1 < b_2$
\item $a_2 < b_1$
\item $a_2 < b_2$
\end{itemize}

Then, there are only 4 segments that could be the result of the intersection $S_1 \cap S_2$:

\begin{itemize}
\item $(a_1, b_1)$
\item $(a_1, b_2)$
\item $(a_2, b_1)$
\item $(a_2, b_2)$
\end{itemize}

Since $p$ is in all of these segments, we have shown that the resulting segment will always contain $p$.
\end{solution}

\begin{solution} %17
\end{solution}

\begin{solution} %18
\end{solution}

\end{document}
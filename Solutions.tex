\documentclass{article}

\usepackage{amsmath}
\usepackage{amsthm}
\usepackage{amssymb}
\theoremstyle{definition}

\newtheorem{solution}{Solution}

\begin{document}
\section*{Problem Sequence - Solutions}
This document will be filled up with the solutions from the problem sequence.

\begin{solution} %1
(Greg)

From Axiom 1, there are three cases to consider:
\begin{enumerate}
\item
$p = 0$

$p$ is equal to the only element of $M$. 
From the definition of limit point, every segment containing $p$ must contain a point of $M$ different from $p$. 
However, there are no points in $M$ different from $p$, so $p$ must not be a limit point of $M$.

\item
$p > 0$

From Axiom 3, there exists a point $a$ such that $0 < a < p$. 
There also exists a point $b$ such that $b > p$. 
Since $p > 0$ and $b > p$, Axiom 2 tells us that $b > 0$. 
We can then form the segment $S = (a, b)$. 
Since $a > 0$ and $b > 0$, $0$ is not between $a$ and $b$, so $S$ does not contain $0$. 
However, $a < p < b$, so $S$ contains $p$. 
$S$ is a segment containing $p$ that does not contain any element of $M$, so $p$ is not a limit point of $M$.

\item
$p < 0$

(symmetric to the $p > 0$ case)
\end{enumerate}

Therefore, regardless of our choice of $p$, we can construct a segment that contradicts the requirements in the limit point definition, so $p$ is not a limit point of $M$.
\end{solution}

\begin{solution} %2
(Jeff)

According to Definition 4, we can prove that $p$ is not a limit point of $M$ if we can construct a segment containing $p$ but not a different point of $M$.
Construct this segment as follows:

\begin{enumerate}
\item
If all points of $M$ are on the opposite side of $p$, choose any value as an endpoint.

\item
If any point of $M$ is on the same side of $p$, choose a point between the nearest point of $M$ and $p$.
(Axiom 3 confirms that there will be such a point.)
\end{enumerate}

This segment contains $p$ but will not contain any points of $M$ (with the exception of $p$, if $p$ is $0$ or $1$).
Therefore, we have found a segment that does not fulfill the requirements of Definition 4, so $p$ is not a limit point of $M$.
\end{solution}

\begin{solution} %3
(Erin, solved after Problem 4)

Let $a = 0$ and $b = 1$. Then, according to Problem 4, $b$ is a limit point of $(a, b)$. Therefore, $1$ is a limit point of $(0,1)$.
\end{solution}

\begin{solution} %4
(Erin)

We will prove that $b$ is a limit point of $(a, b)$; The proof for $a$ is similar.

Construct a segment $(p, q)$ containing $b$ ($p < b < q$).
According to Axiom 1, there are three cases, and we will deal with two of them simultaneously:

\begin{enumerate}
\item $p < a$

In this case, $p < a < b < q$, so $p < (a, b) < q$.
Since $(p, q)$ contains every point of $(a, b)$, we have found points that satisfy Defintion 4.

\item $p \geq a$

In this case, $a \leq p < b < q$.
According to Axiom 3, there is a point $d$ between $p$ and $b$. 
The inequality is then $a \leq p < d < b < q$.
Then,

\begin{itemize}
\item $a < d < b$, so $d$ is in $(a, b)$.
\item $p < d < q$, so $d$ is in $(p, q)$.
\end{itemize}

This means that $(p, q)$ contains $d$, which is a point of $(a, b)$.
\end{enumerate}

Therefore, every $(p, q)$ containing $b$ also contains a point of $(a, b)$, so $b$ is a limit point of $(a, b)$.
\end{solution}

\begin{solution} %5
(Greg)

Choose any point $p$ from $S$.
Construct a segment $(x, y)$ that contains $p$ (ie: $x < p < y$).
Put no other condition on $y$.
According to Axiom 1, one of these three cases is true:

\begin{enumerate}
\item 
$x > a$

Choose a point $q$ between $x$ and $p$ ($x < q < p$; this exists by Axiom 3).
Apply Axiom 2 three times:
\begin{itemize}
\item $a < x$ and $x < q$, so $a < q$
\item $q < p$ and $p < b$, so $q < b$
\item $q < p$ and $p < y$, so $q < y$
\end{itemize}

so $a < q < b$ and $x < q < y$.
Therefore, $q$ is an element of $S$ inside $(a, b)$ that is different from $p$.

\item
$x = a$

Repeat the proof for $x > a$ with one change:

\begin{itemize}
\item $a = x$ and $x < q$, so $a < q$
\end{itemize}

\item
$x < a$

Choose $q$ betweem $a$ and $p$.
Change:

\begin{itemize}
\item $x < a$ and $a < q$, so $x < q$
\end{itemize}

(same conclusion as $x > a$)
\end{enumerate}

In each of these three cases, every $(x,y)$ contains a point $q$ from $(a, b)$.
Therefore, we have satisfied Definition 4, so $p$ is a limit point of $(a, b)$.

\end{solution}

\begin{solution} %6
(Greg)

There are two cases: if $p$ is not a point of $[a, b]$, then either $p < a$ or $p > b$.
We will only consider the first case; the proof for the second follows identically.

Construct two points $x$ and $y$ such that $x < p$ and $p < y < a$ (Axiom 3 guarantees that $y$ exists) and consider the segment $(x, y)$.
Since $x < p < y$, $(x, y)$ contains $p$.

Then, take any point $i$ from $[a, b]$.
This requires that $a \leq i$.
From Axiom 2, since $y < a$ and $a \leq i$, $y < i$.
Then, since $i \nless y$, $(x, y)$ does not contain $i$.

Therefore, $(x, y)$ is a segment containing $p$ but no points of $[a, b]$.
This segment contradicts the definition of a limit point, so $p < a$ is not a limit point of $[a, b]$.
\end{solution}

\begin{solution} %7
(Erin)

If $p$ is a limit point of $H$, then any segment $S$ which contains $p$ also contains a different point of $H$.
Since $H \subseteq K$, this point is also a point of $K$.
Therefore, $S$ also contains a different point of $K$, and $p$ is a limit point of $K$.
\end{solution}

\begin{solution} %8
(Greg)

Consider a point $p$.
According to Axiom 4, there exists a largest integer $M_x$ and a smallest integer $M_y$ such that $M_x < p < M_y$.
Then, choose points $x$ and $y$ from Axiom 3 such that $M_x < x < p$ and $p < y < M_y$, and consider the segment $S = (x, y)$.

We will try to find an integer different from $p$ inside $S$.
Axiom 1 gives us three cases:

\begin{enumerate}
\item $n = p$ (note: this is only possible if $p$ is an integer)

Here, $n$ is not different from $p$, so we have not found an integer different from $p$.

\item $n < p$

Note that $n$ must satisfy $n \leq M_x$; if $n > M_x$, we have contradicted Axiom 4.
Then, from Axiom 2, $n \leq M_x$ and $M_x < x$, so $n < x$.
This shows that $n$ is not between $x$ and $y$, so $S$ does not contain $n$.

\item $n > p$

This case is symmetric to the $n < p$ case.

\end{enumerate}

Therefore, $S$ is a segment containing $p$ that does not contain any integers different from $p$.
Since every element of $M$ is an integer, we have proven that there exists a segment $S$ for every point $p$ without any other points of $M$, so $M$ has no limit points.

\end{solution}

\begin{solution} %9
(Rayne)

Suppose that $S$ does not contain 2 points of $M$. We will find a contradiction.

Let $S$ be the segment $(a, b)$ which contains $p$ (ie: $a < p < b$).  
According to Definition 4, it must also contain a point $m$ of $M$ that is different from $p$.
We will assume that $m < p$ (the case $m > p$ is similar).

We have $a < m < p < b$. 
According to Axiom 3, we can create another segment that contains $p$ but not $m$.
Then, this new segment does not contain any points of $M$, so $p$ must not be a limit point of $M$. 
However, we know that $p$ is a limit point of $M$, so we have a contradiction, and $S$ must contain 2 or more points of $M$.
\end{solution}

\begin{solution} %10
(Greg)

If $S$ is a segment containing $p$, then $S$ contains a point of $K$ different from $p$.
Since every point of $K$ is a limit point of $H$, $S$ contains a limit point of $H$ different from $p$.
According to Problem 9, if $S$ contains a limit point of $H$, $S$ also contains 2 different points of $H$.
If $p$ is equal to one of these points, it will be different from the other one.
Therefore, $S$ contains a point of $H$ different from $p$, satisfying Definition 4, and $p$ is a limit point of $H$.
\end{solution}

\begin{solution} %11
(Greg)

Construct a segment $S = (x, y)$ that contains $0$ ($x < 0 < y$).
Since the reciprocal of a positive number is positive, every element $M_i$ of $M$ is positive, so $x < M_i$ from Axiom 2.

Then, we will attempt to find an element of $M$ that is less than $y$.
Since every element of $M$ is of the form $\frac{1}{n}$ for some integer $n$, we are looking for $\frac{1}{n} < y$.
From the properties of reciprocals, this is equivalent to $\frac{1}{y} < n$.
According to Axiom 4, the point $\frac{1}{y}$ has an integer greater than it, so we can find an $n$ that satisfies this inequality.

Therefore, every $S$ containing $0$ also contains a point of $M$, so $0$ is a limit point of $M$.
\end{solution}

\begin{solution} %12
(Erin)

We will attack this problem in several cases:

\begin{enumerate}
\item $p < 0$

Construct a segment $(a, b)$ that contains both $p$ and 0. 
These points satisfy $a < p < 0 < b$.
From Axiom 3, we can find another point $q$ such that $p < q < 0$.
Then, the segment $(a, q)$ contains $p$, but no points of $M$ (since every point of $M$ is positive), so $p$ must not be a limit point of $M$.

\item $p = 1$

Construct an $(a, b)$ containing $p$ and $\frac{1}{2}$.
The inequality is then $a < \frac{1}{2} < p < b$. 
Axiom 3: can find $q$ such that $\frac{1}{2} < q < p$
Then, $(q, b)$ contains $p$ but no point of $M$ other than $p$ (since every other point of $M$ is less than $\frac{1}{2}$, so $p$ must not be a limit point of $M$.

\item $p > 1$

Construct $(a, b)$ containing $p$ and $1$.
($a < 1 < p < b$)
Axiom 3: can find $q$ such that $1 < q < p$.
Then, $(q, b)$ contains $p$ but no points of $M$.

\item $0 < p < 1$

First, note that if $n$ is an integer, then $n-1 < n < n+1$ (Axiom 5).
Then, $\frac{1}{n+1} < \frac{1}{n} < \frac{1}{n-1}$.
According to Axiom 3, we can find $\frac{1}{a}, \frac{1}{b}$ such that $\frac{1}{n+1} < \frac{1}{a} < \frac{1}{n} < \frac{1}{b} < \frac{1}{n-1}$.
We have two cases from here:

\begin{itemize}
\item $p = \frac{1}{n}$

$(\frac{1}{a}, \frac{1}{b})$ contains $p$ but no other point of $M$, so $p$ is not a limit point of $M$.

\item $\frac{1}{n+1} < p < \frac{1}{n}$

From Axiom 3, we can find $\frac{1}{c}$ such that $p < \frac{1}{c} < \frac{1}{n}$.
Then, $(\frac{1}{a}, \frac{1}{c})$ contains $p$ but no other point of $M$, so $p$ is not a limit point of $M$.
\end{itemize}
\end{enumerate}

These cases cover every point $p \neq 0$. Therefore, if $p$ is not zero, $p$ is not a limit point of $M$.
\end{solution}

\begin{solution} %13
(Erin)

First, note that the set $H \cap K$ is a subset of $H$ because every point of $H \cap K$ must be a point of $H$.
Similarly, $H \cap K \subseteq K$.

Then, from Problem 7, since $p$ is a limit point of $H \cap K$, $p$ is also a limit point of $H$; identically, $p$ is a limit point of $K$.
\end{solution}

\begin{solution} %14
(Jeff)

We will use a counterexample to prove that not every interval containing a limit point must contain another point of the set.

Consider the set $M$ of the reciprocals of all positive integers.
(This was the set discussed in problems 11 and 12.)
We know from Problem 11 that $0$ is a limit point of $M$, so pick the interval $I = [i, 0]$ for any $i < 0$ and let $p = 0$.

Since $p = 0$, which is an endpoint of $I$, $I$ contains $p$.
However, every point $m$ of $M$ is positive ($m > 0$), and is not contained by $I$.
Therefore, $I$ is an interval which contains a limit point of $M$ and no other points of $M$.
This shows that not every interval containing a limit point must contain a different point of the set.
\end{solution}

\begin{solution} %15
(Greg)

Consider an arbitrary segment $S = (S_a, S_b)$ that contains $p$ (so $S_a < p < S_b$).
From Axiom 3, we can find two more points $I_a, I_b$ such that $S_a < I_a < p$ and $p < I_b < S_b$.

Then, we can construct the interval $I = [I_a, I_b]$.
Note that every point of $I$ is also in $S$ - ie, $I \subseteq S$.
Since $I$ contains $p$, it must contain a point of $M$ different from $p$.
Therefore, $S$ also contains a point of $M$ different from $p$, and $p$ is a limit point of $M$.
\end{solution}

\begin{solution} %16
(Fernando)

Let $S_1 = (a_1, b_1)$ and $S_2 = (a_2, b_2)$.
Since $p$ is in both segments, $a_1 < p < b_1$ and $a_2 < p < b_2$.
From Axiom 2, we know that:

\begin{itemize}
\item $a_1 < b_1$
\item $a_1 < b_2$
\item $a_2 < b_1$
\item $a_2 < b_2$
\end{itemize}

Then, there are only 4 segments that could be the result of the intersection $S_1 \cap S_2$:

\begin{itemize}
\item $(a_1, b_1)$
\item $(a_1, b_2)$
\item $(a_2, b_1)$
\item $(a_2, b_2)$
\end{itemize}

Since $p$ is in all of these segments, we have shown that the resulting segment will always contain $p$.
\end{solution}

\begin{solution} %17
(Greg)

$p$ is not a limit point of $H$, so there is a segment $S_H = (H_a, H_b)$ that contains $p$ but no point of $H$ different from $p$.
Likewise, there is a segment $S_K (K_a, K_b)$ that contains $p$ but no different point of $K$.
From these two segments, construct $S = S_H \cap S_K$. 

Since $p$ was in both of the original segments, Problem 16 tells us that $S$ contains $p$.
However, since there were no elements of $H$ or $K$ in these segments (except possibly $p$), $S$ contains no elements of $H$ or $K$.
Finally, because every element of $H \cup K$ is an element of $H$ or $K$, we know that $S$ contains no elements of $H \cup K$.

Therefore, $S$ is a segment that contains $p$ but no other elements of $H \cup K$.
We have found a counterexample to the definition of a limit point, so $p$ is not a limit point of $H \cup K$.
\end{solution}

\begin{solution} %18
(Greg)

We will use a proof by contrapositive. The contrapositive is:

\textit{Show that if $p$ is not a limit point of $H$ and $p$ is not a limit point of $K$, then $p$ is not a limit point of $H \cup K$.}

This is problem 17 (which has been proved), so we are finished.
\end{solution}

\begin{solution} %19
(Erin)

Consider a subset of $M$ that contains $n$ points.
Since this subset has a finite number of points, we can order them $x_0, x_1, x_2, \dots, x_n$ such that $x_0 < x_1 < x_2 < \dots < x_n$.

However, since there is no largest number in $M$, we can find another point $x_{n+1}$ from $M$ such that $x_n < x_{n+1}$.
We can repeat this process indefinitely, so for any positive integer $n$, we can find at least $n$ points in $M$, making $M$ an infinite set.  
\end{solution}

\begin{solution} %20
(Fernando)

Since $M$ is finite, there is a positive integer $n$ such that $M$ does not have $n$ points.
We will then say that the number of points in $M$ is not greater than $n-1$.

If $M$ has only one point, then we can say that its one point is the largest point in $M$.
This is our base case.

Then, if $M$ has more than one point, we can use the following algorithm to find the largest point:

\begin{itemize}
\item
Choose two different points $a_1$ and $a_2$ from $M$.
Then, either $a_1 > a_2$ or $a_1 < a_2$.
Pick the larger of these two points and keep track of it.
(We will assume that $a_1$ was the larger of the two.)

\item
Choose another point $a_3$ from $M$ that is different from $a_1$ and $a_2$.
Then, either $a_3 > a_1$ or $a_3 < a_1$.
If $a_3 > a_1$, from Axiom 2, $a_3 > a_2$, so $a_3$ is the new largest point.
If $a_3 < a_1$, then $a_1$ remains the largest.
\end{itemize}

Since $M$ can have no more than $n-1$ points, we can then find the largest point using $n-2$ of these steps.
Therefore, there must be a largest point of $M$. 
\end{solution}

\begin{solution} %21
(Erin)

According to Problem 9, $S$ must contain at least two points of $M$ different from $p$.
Label these two points $q$ and $s$.
Assume that $q$ is less than $p$ - a similar argument applies if $q > p$.

If $S$ is the segment $(a, b)$ with $a < p < b$, then we can see that $a < q < p$.
Then, we define a new segment $(c, d)$ where $q < c < p$ and $p < d < b$.
This new segment does not contain $q$.
However, it still contains $p$, so there must be another point of $M$ different from $p$ inside $(c, d)$.

Since every point of $(c, d)$ is also in $(a, b)$, the new point that we have found is also in $(a, b)$.
We can repeat this process indefinitely, so $S$ must contain infinitely many points of $M$.
\end{solution}

\begin{solution} %22
\end{solution}

\begin{solution} %23
(Erin)

First, let $n$ be an even positive integer.
Then, $x_n = 0$ and $x_{n+1} = 1$.
If the points $a, b$ satisfy $a < 0 < b < 1$, then the segment $(a, b)$ does not contain $x_{n+1}$.
(Note that this does not depend on $n$.)

Then, let $n$ be an odd positive integer, so $x_n = x_{n+2} = 1$.
The segment $(a, b)$ does not contain $x_{n+2}$.

Therefore, for every positive integer $n$, either $x_{n+1}$ or $x_{n+2}$ is not in $(a, b)$, so we cannot find an $n$ that satisfies Definition 11, and the sequence does not converge to $0$.
\end{solution}

\begin{solution} %24
(no solution required)
\end{solution}

\begin{solution} %25
(Erin)

First, note that `not infinite' means the same thing as `finite'.

Since the sequence is convergent, there is a positive integer $n$ such that $x_n, x_{n+1}, x_{n+2}, \dots$ are all in $(a, b)$.
This does not imply anything about the points $x_1, x_2, x_3, \dots, x_{n-1}$, so these points may or may not be in $(a, b)$.
Counting these points tells us that there cannot be more than $n-1$ points of the sequence outside of $(a, b)$.
Therefore, there are not $n$ points of the sequence not in $(a, b)$, so this set of points is finite. 
\end{solution}

\begin{solution} %26
(Rayne)

Suppose that $n$ is an odd positive integer - the proof for even $n$ is similar.

Since the sequence is defined as
\begin{equation*}
	x_n =
	\begin{cases}
		\frac{1}{n-1}, &n \text{ odd} \\
		\frac{1}{n+1}, &n \text{ even} 
	\end{cases}
\end{equation*}
we can write out several terms of the sequence as

{
\begin{center}
\begin{tabular}{c|c|c|c}
	odd & even & odd & even \\
	$x_n$ & $x_{n+1}$ & $x_{n+2}$ & $x_{n+3}$ \\
	\hline
	$\frac{1}{n+1}$ & $\frac{1}{n}$ & $\frac{1}{n+3}$ & $\frac{1}{n+2}$
\end{tabular}
\end{center}
}

We can see from this sample that $x_{n+2} < x_{n+3} < x_{n} < x_{n+1}$.
Then, if we make a segment $(a,b)$ containing $0$, $x_n$ will be in this segment for large enough $n$.
Since all consecutive terms of the sequence starting at $n+2$ are between $0$ and $b$, then we can see that the definition of convergence is satisfied, and the sequence must converge to $0$.
\end{solution}

\begin{solution} %27
(Greg)

We can find three points $a_1, a_2, a_3$ such that
\begin{equation*}
	a_1 < c < a_2 < d < a_3
\end{equation*}
(Axiom 3 guarantees that $a_2$ exists).
Construct the segments $S_c = (a_1, a_2)$ and $S_d = (a_2, a_3)$.

Notice that any point $p$ in $S_c$ has $p < a_2$ and any point $q$ in $S_d$ has $q > a_2$.
If $p = q$, we contradict Axiom 1, so no point is in both $S_c$ and $S_d$.

The sequence $x_1, x_2, x_3, \dots$ converges to $c$, so there is a positive integer $n$ such that $x_m$ is in $S_c$ for all $m \geq n$.
From above, $x_m$ is not in $S_d$ for all $m \geq n$.
Therefore, we cannot find an $n$ that satisfies Definition 11 for $S_d$, and the sequence does not converge to $d$. 
\end{solution}

\begin{solution} %28
\end{solution}

\begin{solution} %29
\end{solution}

\begin{solution} %30
\end{solution}

\begin{solution} %31
(no solution required)
\end{solution}

\begin{solution} %32
\end{solution}

\begin{solution} %33
(Greg)

We will assume that $M$ has both a rightmost point and a smallest number largest points, and we will find a contradiction.

If the largest point of $M$ is labelled $x$, then for any point $p$ in $M$, $p \leq x$.
Also, if the smallest point greater than every point in $M$ is $y$, then for every $p$ in $M$, $p < y$.
Note that since $x$ is in $M$, $x < y$.

According to Axiom 3, there exists a point $z$ such that $x < z < y$.
Since $x < z$, Axiom 2 tells us that every point of $M$ is less than $z$.
However, $z$ is smaller than $y$, so $y$ is not the smallest point greater than every point of $M$, and we have a contradiction.
\end{solution}

\begin{solution} %34
\end{solution}

\end{document}
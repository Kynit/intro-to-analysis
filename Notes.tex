\documentclass{article}
%\setlength\parindent{0pt}

\usepackage{amsthm}
\usepackage{amssymb}

\theoremstyle{definition}
\newtheorem{axiom}{Axiom}
\newtheorem{definition}{Definition}
\newtheorem{notation}{Notation}

\begin{document}
\section*{Main Course Notes}
There are 5 axioms that we will use in this class:

\begin{axiom}
If each of $a$ and $b$ is a number, then exactly one of the following is true:
\begin{enumerate}
\item
$a = b$
\item
$a < b$
\item
$a > b$
\end{enumerate}
\end{axiom}

\begin{axiom}
If each of $a, b, c$ is a number, $a < b$, and $b < c$, then $a < c$.
\end{axiom}

\begin{axiom}
If $a$ and $b$ are two points on the number line, then there is a point between them.
\end{axiom}

\begin{axiom}
If $a$ is a point, then there is a smallest integer $b$ such that $b > a$ and a largest integer $c$ such that $c < a$.
\end{axiom}

\begin{axiom}
If $n$ is an integer, then $n + 1$ and $n - 1$ are integers, and $n$ is the only integer between $n - 1$ and $n + 1$.
\end{axiom}

Then, as the course continues, we will make various definitions and notation for these definitions.

\begin{definition}
If $a$ and $b$ are two points and $a < b$, the statement that the point $p$ is \textbf{between} the points $a$ and $b$ means that $a < p$ and $p < b$.
\end{definition}

\begin{definition}
A \textbf{point set} is a set of one or more points.
\end{definition}

\begin{definition}
The statement that the point set $S$ is a \textbf{segment} means that there are two points $a$ and $b$, called the \textit{endpoints} of $S$, such that $S$ is the set of all points between $a$ and $b$. 
\end{definition}

\begin{notation}
If $a$ and $b$ are two points and $a < b$ then $(a, b)$ denotes the segment consisting of all points between $a$ and $b$.
\end{notation}
\begin{definition}
If $M$ is a point set and $p$ is a point, the statement that $p$ is a \textbf{limit point} of the point set $M$ means that every segment that contains $p$ contains a point of $M$ different from $p$.
\end{definition}

\begin{definition}
The statement that the point set $I$ is an \textbf{interval} means that there are two points $a$ and $b$, called the \textit{endpoints} of $I$, such that $I$ is the set containing $a$, $b$, and $(a, b)$. $I$ is denoted by $[a, b]$.
\end{definition}
\begin{notation}
If $a$ and $b$ are two points and $a < b$ then $[a, b]$ denotes the interval with endpoints $a$ and $b$.
\end{notation}

\begin{definition}
The statement that the point set $H$ is a \textbf{subset} of the point set $K$ means that if $p$ is a point of $H$, then $p$ is a point of $K$. 
\end{definition}
\begin{notation}
If $H$ is a point set and $K$ is a point set then $H \subseteq K$ means that $H$ is a subset of $K$.
\end{notation}

\begin{definition}
If each of $H$ and $K$ is a point set and there is a point that is in both of them, then the \textbf{intersection} of $H$ and $K$ is the set to which a point $p$ belongs if and only if $p$ is in both $H$ and $K$.
\end{definition}
\begin{notation}
If each of $H$ and $K$ is a point set and there is a point that is in both of them, then $H \cap K$ denotes the intersection of $H$ and $K$.
\end{notation}

\begin{definition}
If each of $H$ and $K$ is a point set, the \textbf{union} of $H$ and $K$ is the set to which the point $p$ belongs if and only if $p$ is in $H$ or $p$ is in $K$.
\end{definition}
\begin{notation}
If each of $H$ and $K$ is a point set, then $H \cup K$ denotes the union of $H$ and $K$. Thus $H \cup K$ is the set of all points in $H$ together with the points in $K$.
\end{notation}

\begin{notation}
If each of $a$, $b$, and $c$ is a number, we will use the notation $M = \{a, b, c\}$ to mean the set containing the points $a$, $b$, and $c$ and no other point. Similarly, we will denote infinite sets, when the pattern is clear, by $M = \{a_1, a_2, a_3, \dots\}$. For example, the set of all positive integers is denoted by $M = \{1, 2, 3, \dots\}$.
\end{notation}

\begin{definition}
The statement that the point set $M$ is \textbf{infinite} means that for every positive integer $n$, $M$ contains at least $n$ points.
\end{definition}

\begin{definition}
The statement that the point set $M$ is \textbf{finite} means that it is not infinite. That is, there is a positive integer $n$ such that $M$ does not contains $n$ points.
\end{definition}

By a \textbf{function} we mean a set of ordered number pairs, no two of which have the same first term. Or, if you prefer, a function is a set of points in the number plane with no two on the same vertical line. If $f$ is a function, then the \textbf{domain} of $f$ is the set of all first terms of ordered pairs of $f$ and the \textbf{range} of $f$ is the set of all second terms of ordered pairs of $f$. By a \textbf{sequence} we mean a function whose domain is the set of positive integers and whose range is a point set.

Usually if $f$ is a function and $(x, y)$ is one of the ordered pairs in $f$, then we denote $y$ by $f(x)$. When $f$ is a sequence and $(n, y)$ is one of the ordered pairs in $f$, then we usually denote $y$ by the short hand $f_n$. Thus we might refer to a sequence by the name of the function as $f$ for example. Or we might refer to a sequence as $f_1, f_2, f_3, \dots$. By a term of a sequence $f$, or a point of (or in) the sequence, we mean $f_n$ for some positive integer $n$.

\begin{definition}
The statement that the sequence $x_1, x_2, x_3, \dots$ \textbf{converges} to the number $c$ or has $c$ as a \textbf{limit} means that for every segment $S$ containing $c$, there is a positive integer $n$ such that each of $x_n, x_{n+1}, x_{n+2}, \dots$ is in $S$. (In other words, for every positive integer $m \geq n$, $x_m$ is in $S$.)
\end{definition}

Note that a sequence is not a point set, but a set of ordered pairs. As such, it does not have a limit point. However, the range of a sequence is a point set and thus might or might not have a limit point.

\begin{definition}
The statement that the point set $M$ is \textbf{bounded above} means that there is a number $c$ such that each point of $M$ is to the left of $c$.
\end{definition}

\begin{definition}
The statement that the point set $M$ is \textbf{bounded below} means that there is a number $c$ such that each point of $M$ is to the right of $c$.
\end{definition}

\end{document}
\documentclass{article}
%\setlength\parindent{0pt}
\newtheorem{solution}{Solution}

\begin{document}
\section*{Problem Sequence - Solutions}
This document will be filled up with the solutions from the problem sequence.

\begin{solution} %1
From Axiom 1, there are three cases to consider:
\begin{enumerate}
\item
$p = 0$

$p$ is equal to the only element of $M$. From the definition of limit point, every segment containing $p$ must contain a point of $M$ different from $p$. However, there are no points in $M$ different from $p$, so $p$ must not be a limit point of $M$.

\item
$p > 0$

From Axiom 3, there exists a point $a$ such that $0 < a < p$. There also exists a point $b$ such that $b > p$. Since $p > 0$ and $b > p$, Axiom 2 tells us that $b > 0$. We can then form the segment $S = (a, b)$. Since $a > 0$ and $b > 0$, $0$ is not between $a$ and $b$, so $S$ does not contain $0$. However, $a < p < b$, so $S$ contains $p$. $S$ is a segment containing $p$ that does not contain any element of $M$, so $p$ is not a limit point of $M$.

\item
$p < 0$

(symmetric to the $p > 0$ case)
\end{enumerate}

Therefore, regardless of our choice of $p$, we can construct a segment that contradicts the requirements in the limit point definition, so $p$ is not a limit point of $M$.
\end{solution}

\begin{solution}
\end{solution}

\begin{solution}
\end{solution}

\begin{solution}
\end{solution}

\begin{solution} %5
\end{solution}

\begin{solution} %6
\end{solution}

\begin{solution} %7
\end{solution}

\begin{solution}
\end{solution}

\begin{solution}
\end{solution}

\begin{solution} %10
According to Axiom 4, there is a largest integer $a$ such that $a < p$ and a smallest integer $b$ such that $p < b$. Then, using Axiom 3, we find a point $x$ between $a$ and $p$ and a point $y$ between $p$ and $b$ ($a < x < p$ and $p < y < b$). From Axiom 2, $x < b$ and $a < y$.
\begin{enumerate}
\item
$p$ is an integer

\item
$p$ is not an integer

\end{enumerate}
(need to think about these)
\end{solution}

\begin{solution}
\end{solution}

\begin{solution}
\end{solution}

\begin{solution}
\end{solution}

\begin{solution}
\end{solution}

\begin{solution}
\end{solution}

\begin{solution}
\end{solution}

\begin{solution}
\end{solution}

\begin{solution}
\end{solution}

\end{document}
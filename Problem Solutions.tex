\documentclass{article}
%\setlength\parindent{0pt}
\newtheorem{solution}{Solution}

\begin{document}
\section*{Problem Sequence - Solutions}
This document will be filled up with the solutions from the problem sequence.

\begin{solution} 
(Greg)

From Axiom 1, there are three cases to consider:
\begin{enumerate}
\item
$p = 0$

$p$ is equal to the only element of $M$. 
From the definition of limit point, every segment containing $p$ must contain a point of $M$ different from $p$. 
However, there are no points in $M$ different from $p$, so $p$ must not be a limit point of $M$.

\item
$p > 0$

From Axiom 3, there exists a point $a$ such that $0 < a < p$. 
There also exists a point $b$ such that $b > p$. 
Since $p > 0$ and $b > p$, Axiom 2 tells us that $b > 0$. 
We can then form the segment $S = (a, b)$. 
Since $a > 0$ and $b > 0$, $0$ is not between $a$ and $b$, so $S$ does not contain $0$. 
However, $a < p < b$, so $S$ contains $p$. 
$S$ is a segment containing $p$ that does not contain any element of $M$, so $p$ is not a limit point of $M$.

\item
$p < 0$

(symmetric to the $p > 0$ case)
\end{enumerate}

Therefore, regardless of our choice of $p$, we can construct a segment that contradicts the requirements in the limit point definition, so $p$ is not a limit point of $M$.
\end{solution}

\begin{solution} %2
(Jeff)

According to Definition 4, we can prove that $p$ is not a limit point of $M$ if we can construct a segment containing $p$ but not a different point of $M$.
Construct this segment as follows:

\begin{enumerate}
\item
If all points of $M$ are on the opposite side of $p$, choose any value as an endpoint.

\item
If any point of $M$ is on the same side of $p$, choose a point between the nearest point of $M$ and $p$.
(Axiom 3 confirms that there will be such a point.)
\end{enumerate}

This segment contains $p$ but will not contain any points of $M$ (with the exception of $p$, if $p$ is $0$ or $1$).
Therefore, we have found a segment that does not fulfill the requirements of Definition 4, so $p$ is not a limit point of $M$.
\end{solution}

\begin{solution} %3
\end{solution}

\begin{solution} %4
(Erin)
\end{solution}

\begin{solution} %5
(Greg)

Choose any point $p$ from $S$.
Construct a segment $(x, y)$ that contains $p$ (ie: $x < p < y$).
Put no other condition on $y$.
According to Axiom 1, one of these three cases is true:

\begin{enumerate}
\item 
$x > a$

Choose a point $q$ between $x$ and $p$ ($x < q < p$; this exists by Axiom 3).
Apply Axiom 2 three times:
\begin{itemize}
\item $a < x$ and $x < q$, so $a < q$
\item $q < p$ and $p < b$, so $q < b$
\item $q < p$ and $p < y$, so $q < y$
\end{itemize}

so $a < q < b$ and $x < q < y$.
Therefore, $q$ is an element of $S$ inside $(a, b)$ that is different from $p$.

\item
$x = a$

Repeat the proof for $x > a$ with one change:

\begin{itemize}
\item $a = x$ and $x < q$, so $a < q$
\end{itemize}

\item
$x < a$

Choose $q$ betweem $a$ and $p$.
Change:

\begin{itemize}
\item $x < a$ and $a < q$, so $x < q$
\end{itemize}

(same conclusion as $x > a$)
\end{enumerate}

In each of these three cases, every $(x,y)$ contains a point $q$ from $(a, b)$.
Therefore, we have satisfied Definition 4, so $p$ is a limit point of $(a, b)$.

\end{solution}

\begin{solution} %6
(Zack)
\end{solution}

\begin{solution} %7
(Amber)

\end{solution}

\begin{solution} %8
(Greg)

\end{solution}

\begin{solution}
\end{solution}

\begin{solution} %10
\end{solution}

\begin{solution}
\end{solution}

\begin{solution}
\end{solution}

\begin{solution}
\end{solution}

\begin{solution}
\end{solution}

\begin{solution}
\end{solution}

\begin{solution}
\end{solution}

\begin{solution}
\end{solution}

\begin{solution}
\end{solution}

\end{document}